%&latex
\documentclass[10pt]{article}
\makeatletter
\renewcommand{\@seccntformat}[1]{}
\makeatother

\usepackage[utf8]{inputenc}
\usepackage{amssymb}
\usepackage{amsmath}
\usepackage{geometry}
\usepackage{listings}
\usepackage{xcolor}

\definecolor{codegreen}{rgb}{0,0.6,0}
\definecolor{codegray}{rgb}{0.5,0.5,0.5}
\definecolor{codepurple}{rgb}{0.58,0,0.82}
\definecolor{backcolour}{rgb}{0.95,0.95,0.92}

\lstdefinestyle{mystyle}{
	backgroundcolor=\color{backcolour},
	commentstyle=\color{codegreen},
	keywordstyle=\color{magenta},
	numberstyle=\tiny\color{codegray},
	stringstyle=\color{codepurple},
	basicstyle=\ttfamily\footnotesize,
	breakatwhitespace=false,
	breaklines=true,
	captionpos=b,
	keepspaces=true,
	numbers=left,
	numbersep=5pt,
	showspaces=false,
	showstringspaces=false,
	showtabs=false,
	tabsize=2,
    numbers=none
}

\lstset{style=mystyle}

\begin{document}
\begin{center}
    \Huge { \textbf{A: 5 Dinar Coins}} \\ 
    \normalsize  { input:  STDIN}\\
    \normalsize{    output: STDOUT}
\end{center}
\section{Problem Statement.}
Amir got paid big time from ACM INSAT and now has K 5-Dinar coins.\\
If these coins add up to S Dinars or more he is considered Rich. Your job is to find out if Amir is rich or not.
\paragraph{}
\section{Input.}
$$ 1\le K \le 100 $$
$$ 1\le S \le 100,000 $$
\section{Output.}

If the coins add up to S Dinar or more, print “Rich!”; otherwise, print “Not yet”

\section{Examples.}
\subsection{example 1}
Input:
\begin{lstlisting}[language=Python]
2 9
\end{lstlisting}
Output:
\begin{lstlisting}[language=Python]
Rich!
\end{lstlisting}
\newpage
\begin{center}
    \Huge { \textbf{B: Triple Dots}}\\
    \normalsize  { input:  STDIN}\\
    \normalsize{    output: STDOUT}

\end{center}

\section{Problem Statement.}
\paragraph{}
Sondes is in charge of the recruitement process in her company. She starts by registering the names of the new members.
However, because of the large number of these members, she decided to cut some corners.
\paragraph{}
When a new member with the name S decides to join, if the length of his name is less than or equal to K, she will write out his/her full name, but if the name is longer than K, she will write the first K characters of the name and append three dots (...).\\
For example, when K = 3 and she receives the name “Ahmed” she will register him as “Ahm...”\\
And when K = 5 and she receives the name “Aycha” she will register her as “Aycha”\\
\paragraph{}
Given K and the name S, find out how Sondes will register the name S.
\paragraph{}
\section{Constraints.}
$$ 1\le K \le 100 $$
$$ 1\le length(S) \le 100 $$
\section{Output.}
Print the string as stated in the problem statement.
\section{Examples.}
\subsection{example 1}
Input:
\begin{lstlisting}[language=Python]
3 Ahmed
\end{lstlisting}
Output:
\begin{lstlisting}[language=Python]
Ahm...
\end{lstlisting}
\newpage
\begin{center}
    \Huge { \textbf{C: Ons and factorials}}\\
    \normalsize  { input:  STDIN}\\
    \normalsize{    output: STDOUT}
\end{center}
\section{Problem Statement.}
Ons likes to count the number of consecutive zeros at the end of factorials\\
Given an integer N , return the number of trailing zeroes in N! 
$$ n!= n*(n-1)*(n-2)...1 $$
e.g: $ 5!=5*4*3*2*1=120 $
\paragraph{}
\section{Input}
$$ 1\le N \le 100000 $$
\section{Output.}
the number of trailing zeroes in N!.
\section{Examples.}
\subsection{example 1}
Input:
\begin{lstlisting}[language=Python]
3
\end{lstlisting}
Output:
\begin{lstlisting}[language=Python]
0
\end{lstlisting}

\subsection{example 2}
Input:
\begin{lstlisting}[language=Python]
5
\end{lstlisting}
Output:
\begin{lstlisting}[language=Python]
1
\end{lstlisting}
\textbf{Explanation 1:} 3! = 6, no trailing zero.\\
\textbf{Explanation 2:} 5! = 120, one trailing zero.
\newpage
\begin{center}
    \Huge { \textbf{D: BAD 7!}}\\
    \normalsize  { input:  STDIN}\\
    \normalsize{    output: STDOUT}
\end{center}
\section{Problem Statement.}
You hate the number 7 right? Me too!\\
Let us count the number of integers without the digit 7 in both \textbf{decimal} (base 10) and \textbf{octal} (base 8).\\
How many such integers are there between 1 and N?
\section{Input}
$$ 1\le N \le 100 $$
\section{Output.}
Print an integer representing the answer.
\section{Examples.}
\subsection{example 1}
Input:
\begin{lstlisting}[language=Python]
20
\end{lstlisting}
Output:
\begin{lstlisting}[language=Python]
17
\end{lstlisting}
Among the integers between 1 and 20, 7 and 17 contain the digit 7 in decimal. Also , 7 and 15 contain the digit 7 in octal.\\
And so , the 17 integers other than 7, 15, and 17 meet the requirements.
\newpage
\begin{center}
    \Huge { \textbf{E: Only One Clue}}\\
    \normalsize  { input:  STDIN}\\
    \normalsize{    output: STDOUT}
\end{center}
\section{Problem Statement.}
\paragraph{}
There are 201 stones placed on a line. The coordinates of these stones are -100, -99, -98, ..... -1, 0, 1, 2, .... 99, 100\\
Among them, only K consecutive stones are painted black, while the others are painted white.\\
We want you to find out which of the stones can potentially be black, and we’ll give you one clue.\\
The clue is that the stone with the number X is black!\\
Print all the coordinates of the stones that potentially can be painted black, in ascending order.
\section{Input}
$$ 1\le K \le 50 $$
$$ -20\le X \le 20 $$
\section{Output.}
Print all the coordinates of the stones that potentially can be painted black, in ascending order, with spaces in between.
\section{Examples.}
\subsection{example 1}
Input:
\begin{lstlisting}[language=Python]
3 7
\end{lstlisting}
Output:
\begin{lstlisting}[language=Python]
5 6 7 8 9
\end{lstlisting}
We know that there are three stones painted black, and the stone at coordinate 7 is painted black. There are three possible cases:
\begin{itemize}
\item The three stones painted black are placed at coordinates 5 , 6, and 7.
\item The three stones painted black are placed at coordinates 6, 7, and 8.
\item The three stones painted black are placed at coordinates 7, 8, and 9.
\end{itemize}
Thus, five coordinates potentially contain a stone painted black: 5 , 6, 7, 8, and 9.
\newpage
\begin{center}
    \Huge { \textbf{F: The Ink Problem}}\\
\normalsize  { input:  STDIN}\\
\normalsize{    output: STDOUT}
\end{center}
\section{Problem Statement.}
Yagami Light is the protagonist of a popular anime show named “Death Note”.\\
In order to prove his ideals to the world, Yagami decided to eradicate every criminal that exists by using
his death note (which is a book used to kill people).\\
He plans to use N notes where each note consumes I ink. Initially, He has an ink limit of L.\\
Help him find the maximum number of notes he can write.
\section{Constraints}
$$ 1\le N \le 100 $$
$$ 10 \le L \le 200 $$
\section{Input}
The first line contains two integers, the size of the array N and the ink limit L.\\
The second line contains N integers $I_0 I_1 I_2 I_3 ... I_{N-1} $ ($I_j$ corresponds to the $j_{th}$ element in the array
where $ 1\le I_j \le 200 $).
\section{Output.}
Print the maximum number of notes Yagami can write.
\section{Examples.}
\subsection{example 1}
Input:
\begin{lstlisting}[language=Python]
9 190
10 18 71 22 17 180 52 192 99
\end{lstlisting}
Output:
\begin{lstlisting}[language=Python]
6
\end{lstlisting}

\subsection{example 2}
Input:
\begin{lstlisting}[language=Python]
1 140
142
\end{lstlisting}
Output:
\begin{lstlisting}[language=Python]
0
\end{lstlisting}
\newpage
\begin{center}
    \Huge { \textbf{G:That's Magic}}\\
\normalsize  { input:  STDIN}\\
\normalsize{    output: STDOUT}
\end{center}

\section{Problem Statement.}
A magician has the following three cards:
\begin{itemize}
\item A red card with the integer A.
\item A green card with the integer B.
\item A blue card with the integer C.
\end{itemize}
He can do the following operation \textbf{at most K times} (K times or less)
\begin{itemize}
\item Choose one of the three cards and multiply the written integer on it by 2.
\end{itemize}
His magic trick is successful if, after the operations, these conditions are satisfied :
\begin{itemize}
\item The integer on the green card is \textbf{strictly} greater than the integer on the red card.
\item The integer on the blue card is \textbf{strictly} greater than the integer on the green card.
\end{itemize}
Can the magician perform his trick successfully?\\
If the magic can be successful, print “You tricked us!”; otherwise, print ”Oh no!”

\section{Input}
$$ 1\le A \le 10 $$
$$ 1\le B \le 10 $$
$$ 1\le C \le 10 $$
$$ 1\le K \le 20 $$
\section{Output.}
If the magic can be successful, print “You tricked us!”; otherwise, print ”Oh no!”
\section{Examples.}
\subsection{example 1}
Input:
\begin{lstlisting}[language=Python]
7 2 5 3 
\end{lstlisting}
Output:
\begin{lstlisting}[language=Python]
You tricked us!
\end{lstlisting}
The magic will be successful if, for example, he does the following operations:
\begin{itemize}
\item First, choose the blue card. The integers on the red, green, and blue cards are now 7, 2, and 10, respectively.
\item Second, choose the green card. The integers on the red, green, and blue cards are now 7, 4, and 10, respectively.
\item Third, choose the green card. The integers on the red, green, and blue cards are now 7, 8, and 10, respectively.
\end{itemize}
\newpage
\begin{center}
    \Huge { \textbf{H: Meriem's Magic}}\\
\normalsize  { input:  STDIN}\\
\normalsize{    output: STDOUT}
\end{center}

\section{Problem Statement.}
\paragraph{}
In her quest to rule the world, Meriem now have to face the lord of the dark world.
In order to defeat this monster Meriem must find K prime numbers that sum to the
magic number N. Meriem remembered Goldbach's conjecture and thought that it could help her.
\paragraph{}
your task now is to help her find K primes that sum to N.
\paragraph{}
\textbf{definition of a prime.}
A prime number (or prime) is a natural number greater than 1 that has no positive divisors other than 1 and itself
(like 2,3,5,7,11 ...)
\paragraph{}
\textbf{Goldbach's conjecture.} states that any even number (greater than 2) is a sum of two primes.
$$ \forall\ 2<n\ \ \exists\  p1,p2\ such\ that\  p1+p2=n $$
where n is even and p1 , p2 are two primes. 
this conjecture have been proven to hold for any even number (greater than 2) less than $10^{18}$
\paragraph{}
can you help meriem find K primes such that their sum = N, or say that they dont exist?
\section{Input.}
the only input will be the magic number n and k the number of primes in that order.
$$4\le N \le 2500\ \  and\ \  2 \le K \le 1000.$$

\section{Output.}

print out K prime numbers that sum to N. or if this is not possible print "Impossible" without the quotes.

\section{Examples.}
\subsection{example 1}
Input:
\begin{lstlisting}[language=Python]
2036 4
\end{lstlisting}
Output:
\begin{lstlisting}[language=Python]
509 509 509 509 
\end{lstlisting}
\subsection{example 2}
Input:
\begin{lstlisting}[language=Python]
1473 3
\end{lstlisting}
Output:
\begin{lstlisting}[language=Python]
13 541 919
\end{lstlisting}
\subsection{example 3}
Input:
\begin{lstlisting}[language=Python]
1000 1000
\end{lstlisting}
Output:
\begin{lstlisting}[language=Python]
Impossible
\end{lstlisting}
\subsection{example 4}
Input:
\begin{lstlisting}[language=Python]
15 3
\end{lstlisting}
Output:
\begin{lstlisting}[language=Python]
3 7 5
\end{lstlisting}



\newpage
\begin{center}
    \Huge { \textbf{I: Blue Cells}}\\
\normalsize  { input:  STDIN}\\
\normalsize{    output: STDOUT}
\end{center}
\section{Problem Statement.}
\paragraph{}
There is a 2-D matrix of R rows and C columns, all of its cells are white initially.
\paragraph{}
X will choose x of the rows and y of the columns, and paint all of the cells contained in those rows or columns with blue paint. When a row is painted, all of its cells are painted (same for a column).\\
Now X is wondering; how many white cells will remain?
\section{Input}
$$ 1\le R \le 100 $$
$$ 1\le C \le 100 $$
$$ 0\le x \le R $$
$$ 0\le y \le C $$
\section{Output.}
Print the number of white cells that will remain.
\section{Examples.}
\subsection{example 1}
Input:
\begin{lstlisting}[language=Python]
3 2
2 1
\end{lstlisting}
Output:
\begin{lstlisting}[language=Python]
1
\end{lstlisting}
The matrix has 3 rows and 2 columns(6 cells in total).\\
X painted 2 rows and 1 column in blue, so only 1 cell will remain white.
\newpage
\begin{center}
    \Huge { \textbf{J: 404}}\\
\normalsize  { input:  STDIN}\\
\normalsize{    output: STDOUT}
\end{center}
\section{Problem Statement.}
There exists n bags of gold each of which contains $b_i$ golden coins ($1 \le i \le
 n$) Nawres has to collect at least S golden coins but she does not have much
 time.\\
 so she decided to pick k consecutive bags starting from a random index j
 ($1 \le j \le n - k +1$).\\
 Help Nawres find the minimum integer k so that whatever index j she chooses it is guaranteed that the sum of coins she will collect ($b_j
 + ... + b_{j+k-1}$) will be greater than or equal to S.\\
 if Nawres can never collect more than the sum S print "impossible"
\section{Input.}
you will be given as input:
n ($1<=n<=10^5$) the number of bags.\\
S ($1<=S<=10^6$) the amount of coins she has to collect.\\
and n numbers $p_i$ ($0<=p_i<=10^5$) representing the number of coins in the i-th bag.
\section{Output.}
print k if it exists and "impossible" if it doesn't exist.

\section{Examples.}
\subsection{example 1}
Input:
\begin{lstlisting}[language=Python]
8 5
1 0 4 5 0 0 2 1
\end{lstlisting}
Output:
\begin{lstlisting}[language=Python]
5
\end{lstlisting}
\subsection{example 2}
Input:
\begin{lstlisting}[language=Python]
6 1
1 1 1 1 1 1
\end{lstlisting}
Output:
\begin{lstlisting}[language=Python]
1
\end{lstlisting}
\subsection{example 3}
Input:
\begin{lstlisting}[language=Python]
4 10
3 1 1 2
\end{lstlisting}
Output:
\begin{lstlisting}[language=Python]
impossible
\end{lstlisting}




\end{document}
