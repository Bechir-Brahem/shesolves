%&latex
\documentclass[10pt]{article}
\makeatletter
\renewcommand{\@seccntformat}[1]{}
\makeatother

\usepackage[utf8]{inputenc}
\usepackage{amssymb}
\usepackage{amsmath}
\usepackage{geometry}
\usepackage{listings}
\usepackage{xcolor}

\definecolor{codegreen}{rgb}{0,0.6,0}
\definecolor{codegray}{rgb}{0.5,0.5,0.5}
\definecolor{codepurple}{rgb}{0.58,0,0.82}
\definecolor{backcolour}{rgb}{0.95,0.95,0.92}

\lstdefinestyle{mystyle}{
	backgroundcolor=\color{backcolour},
	commentstyle=\color{codegreen},
	keywordstyle=\color{magenta},
	numberstyle=\tiny\color{codegray},
	stringstyle=\color{codepurple},
	basicstyle=\ttfamily\footnotesize,
	breakatwhitespace=false,
	breaklines=true,
	captionpos=b,
	keepspaces=true,
	numbers=left,
	numbersep=5pt,
	showspaces=false,
	showstringspaces=false,
	showtabs=false,
	tabsize=2,
    numbers=none
}

\lstset{style=mystyle}

\begin{document}
    \Huge { \textbf{DEPRECATED VERSION}}\\

\begin{center}
    \Huge { \textbf{A: Math is life}}\\
    \normalsize  { input:  STDIN}\\
    \normalsize{    output: STDOUT}
\end{center}
\section{Problem Statement.}
\paragraph{}
Welcome to shesolves!
Let’s start by testing your advanced mathematical skills, shall we?
\paragraph{}
We have two integers: A and B, can you find the largest number among A+B, A-B, A*B ?
\paragraph{}
\section{Input}
$$ -100\le A \le 100 $$
$$ -100\le B \le 100 $$
\section{Output.}
Print the largest number among A+B, A-B, A*B
\section{Examples.}
\subsection{example 1}
Input:
\begin{lstlisting}[language=Python]
-13 3
\end{lstlisting}
Output:
\begin{lstlisting}[language=Python]
-10
\end{lstlisting}
\newpage
\begin{center}
    \Huge{    \textbf{B: Siraje's OCD}}\\
        \normalsize  { input:  STDIN}\\
    \normalsize{    output: STDOUT}
\end{center}
\section{Problem Statement.}
Sirajeddine is a victim of Obsessive-Compulsive Disorder (OCD). He wants everything to be always tidy(incuding arrays). So when he sees an array that is not completely sorted, his OCD activates !\\
\\
Let’s define an “intrusion” as an element that is not in its correct place in the sorted version of the array.\\
Given N and an array A of numbers of length N, return the number of intrusions.For example, if N = 3 and the array A = [1, 3, 2]:\\
The sorted version of the array A will be [1, 2, 3] and thus the number of intrusions will be equal to 2. Because the elements 2 and 3 are not in their correct place.
\paragraph{}
\section{Input}
$$ 1\le N \le 100 $$
$$ 1\le A_i \le N $$
\section{Output.}
Print the number of intrusions in the array.
\section{Examples.}
\subsection{example 1}
Input:
\begin{lstlisting}[language=Python]
3
1 3 2
\end{lstlisting}
Output:
\begin{lstlisting}[language=Python]
2
\end{lstlisting}

\end{document}

