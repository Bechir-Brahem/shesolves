%&latex
\documentclass[10pt]{article}
\makeatletter
\renewcommand{\@seccntformat}[1]{}
\makeatother
\usepackage[utf8]{inputenc}
\usepackage{amssymb}
\usepackage{amsmath}
\usepackage{geometry}
\usepackage{listings}
\usepackage{xcolor}

\definecolor{codegreen}{rgb}{0,0.6,0}
\definecolor{codegray}{rgb}{0.5,0.5,0.5}
\definecolor{codepurple}{rgb}{0.58,0,0.82}
\definecolor{backcolour}{rgb}{0.95,0.95,0.92}

\lstdefinestyle{mystyle}{
	backgroundcolor=\color{backcolour},
	commentstyle=\color{codegreen},
	keywordstyle=\color{magenta},
	numberstyle=\tiny\color{codegray},
	stringstyle=\color{codepurple},
	basicstyle=\ttfamily\footnotesize,
	breakatwhitespace=false,
	breaklines=true,
	captionpos=b,
	keepspaces=true,
	numbers=left,
	numbersep=5pt,
	showspaces=false,
	showstringspaces=false,
	showtabs=false,
	tabsize=2,
    numbers=none
}

\lstset{style=mystyle}

\begin{document}
\title{5 Dinar Coins}
 \date{}
\maketitle
\section{Problem Statement.}
Amir got paid big time from ACM INSAT and now has K 5-Dinar coins.\\
If these coins add up to S Dinars or more he is considered Rich. Your job is to find out if Amir is rich or not.
\paragraph{}
\section{Input.}
$$ 1\le K \le 100 $$
$$ 1\le S \le 100,000 $$
\section{Output.}

If the coins add up to S Dinar or more, print “Rich”; otherwise, print “Not yet”

\section{Examples.}
\subsection{example 1}
Input:
\begin{lstlisting}[language=Python]
2 9
\end{lstlisting}
Output:
\begin{lstlisting}[language=Python]
Rich!
\end{lstlisting}
\end{document}

