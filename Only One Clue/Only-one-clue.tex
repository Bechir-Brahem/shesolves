%&latex
\documentclass[10pt]{article}
\makeatletter
\renewcommand{\@seccntformat}[1]{}
\makeatother
\usepackage[utf8]{inputenc}
\usepackage{amssymb}
\usepackage{amsmath}
\usepackage{geometry}
\usepackage{listings}
\usepackage{xcolor}

\definecolor{codegreen}{rgb}{0,0.6,0}
\definecolor{codegray}{rgb}{0.5,0.5,0.5}
\definecolor{codepurple}{rgb}{0.58,0,0.82}
\definecolor{backcolour}{rgb}{0.95,0.95,0.92}

\lstdefinestyle{mystyle}{
	backgroundcolor=\color{backcolour},
	commentstyle=\color{codegreen},
	keywordstyle=\color{magenta},
	numberstyle=\tiny\color{codegray},
	stringstyle=\color{codepurple},
	basicstyle=\ttfamily\footnotesize,
	breakatwhitespace=false,
	breaklines=true,
	captionpos=b,
	keepspaces=true,
	numbers=left,
	numbersep=5pt,
	showspaces=false,
	showstringspaces=false,
	showtabs=false,
	tabsize=2,
    numbers=none
}

\lstset{style=mystyle}

\begin{document}
\title{Only One Clue}
 \date{}
\maketitle
\section{Problem Statement}
\paragraph{}
There are 201 stones placed on a line. The coordinates of these stones are -100, -99, -98, ..... -1, 0, 1, 2, .... 99, 100\\
Among them, only K consecutive stones are painted black, while the others are painted white.\\
We want you to find out which of the stones can potentially be black, and we’ll give you one clue.\\
The clue is that the stone with the number X is black!\\
Print all the coordinates of the stones that potentially can be painted black, in ascending order.
\section{Input}
$$ 1\le K \le 50 $$
$$ -20\le X \le 20 $$
\section{Output}
Print all the coordinates of the stones that potentially can be painted black, in ascending order, with spaces in between.
\section{Examples}
\subsection{Example 1}
Input:
\begin{lstlisting}[language=Python]
3 7
\end{lstlisting}
Output:
\begin{lstlisting}[language=Python]
5 6 7 8 9
\end{lstlisting}
We know that there are three stones painted black, and the stone at coordinate 7 is painted black. There are three possible cases:
\begin{itemize}
\item The three stones painted black are placed at coordinates 5 , 6, and 7.
\item The three stones painted black are placed at coordinates 6, 7, and 8.
\item The three stones painted black are placed at coordinates 7, 8, and 9.
\end{itemize}
Thus, five coordinates potentially contain a stone painted black: 5 , 6, 7, 8, and 9.


\end{document}

