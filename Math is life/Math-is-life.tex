
%&latex
\documentclass[10pt]{article}
\usepackage[utf8]{inputenc}
\usepackage{amssymb}
\usepackage{amsmath}
\usepackage{geometry}
\usepackage{listings}
\usepackage{xcolor}

\definecolor{codegreen}{rgb}{0,0.6,0}
\definecolor{codegray}{rgb}{0.5,0.5,0.5}
\definecolor{codepurple}{rgb}{0.58,0,0.82}
\definecolor{backcolour}{rgb}{0.95,0.95,0.92}

\lstdefinestyle{mystyle}{
	backgroundcolor=\color{backcolour},
	commentstyle=\color{codegreen},
	keywordstyle=\color{magenta},
	numberstyle=\tiny\color{codegray},
	stringstyle=\color{codepurple},
	basicstyle=\ttfamily\footnotesize,
	breakatwhitespace=false,
	breaklines=true,
	captionpos=b,
	keepspaces=true,
	numbers=left,
	numbersep=5pt,
	showspaces=false,
	showstringspaces=false,
	showtabs=false,
	tabsize=2,
    numbers=none
}

\lstset{style=mystyle}

\begin{document}
\title{Math is life}
 \date{}
\maketitle
\section{Problem Statement.}
\paragraph{}
Welcome to shesolves!
Let’s start by testing your advanced mathematical skills, shall we?
\paragraph{}
We have two integers: A and B, can you find the largest number among A+B, A-B, A*B ?
\paragraph{}
\section{Input}
$$ -100\le A \le 100 $$
$$ -100\le B \le 100 $$
\section{Output.}
Print the largest number among A+B, A-B, A*B
\section{Examples.}
\subsection{example 1}
Input:
\begin{lstlisting}[language=Python]
-13 3
\end{lstlisting}
Output:
\begin{lstlisting}[language=Python]
-10
\end{lstlisting}
\end{document}

