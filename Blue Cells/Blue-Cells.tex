


%&latex
\documentclass[10pt]{article}
\usepackage[utf8]{inputenc}
\usepackage{amssymb}
\usepackage{amsmath}
\usepackage{geometry}
\usepackage{listings}
\usepackage{xcolor}

\definecolor{codegreen}{rgb}{0,0.6,0}
\definecolor{codegray}{rgb}{0.5,0.5,0.5}
\definecolor{codepurple}{rgb}{0.58,0,0.82}
\definecolor{backcolour}{rgb}{0.95,0.95,0.92}

\lstdefinestyle{mystyle}{
	backgroundcolor=\color{backcolour},
	commentstyle=\color{codegreen},
	keywordstyle=\color{magenta},
	numberstyle=\tiny\color{codegray},
	stringstyle=\color{codepurple},
	basicstyle=\ttfamily\footnotesize,
	breakatwhitespace=false,
	breaklines=true,
	captionpos=b,
	keepspaces=true,
	numbers=left,
	numbersep=5pt,
	showspaces=false,
	showstringspaces=false,
	showtabs=false,
	tabsize=2,
    numbers=none
}

\lstset{style=mystyle}

\begin{document}
\title{Blue Cells}
 \date{}
\maketitle
\section{Problem Statement.}
\paragraph{}
There is a 2-D matrix of R rows and C columns, all of its cells are white initially.
\paragraph{}
X will choose x of the rows and y of the columns, and paint all of the cells contained in those rows or columns with blue paint. When a row is painted, all of its cells are painted (same for a column).\\
Now X is wondering; how many white cells will remain?
\section{Input}
$$ 1\le R \le 100 $$
$$ 1\le C \le 100 $$
$$ 0\le x \le R $$
$$ 0\le y \le C $$
\section{Output.}
Print the number of white cells that will remain.
\section{Examples.}
\subsection{example 1}
Input:
\begin{lstlisting}[language=Python]
3 2
2 1
\end{lstlisting}
Output:
\begin{lstlisting}[language=Python]
1
\end{lstlisting}
The matrix has 3 rows and 2 columns(6 cells in total).\\
X painted 2 rows and 1 column in blue, so only 1 cell will remain white.


\end{document}

