%&latex
\documentclass[10pt]{article}
\makeatletter
\renewcommand{\@seccntformat}[1]{}
\makeatother
\usepackage[utf8]{inputenc}
\usepackage{amssymb}
\usepackage{amsmath}
\usepackage{geometry}
\usepackage{listings}
\usepackage{xcolor}

\definecolor{codegreen}{rgb}{0,0.6,0}
\definecolor{codegray}{rgb}{0.5,0.5,0.5}
\definecolor{codepurple}{rgb}{0.58,0,0.82}
\definecolor{backcolour}{rgb}{0.95,0.95,0.92}

\lstdefinestyle{mystyle}{
	backgroundcolor=\color{backcolour},
	commentstyle=\color{codegreen},
	keywordstyle=\color{magenta},
	numberstyle=\tiny\color{codegray},
	stringstyle=\color{codepurple},
	basicstyle=\ttfamily\footnotesize,
	breakatwhitespace=false,
	breaklines=true,
	captionpos=b,
    numbers=none,
	keepspaces=true,
	numbers=left,
	numbersep=5pt,
	showspaces=false,
	showstringspaces=false,
	showtabs=false,
	tabsize=2
}

\lstset{style=mystyle}

\begin{document}
%+Title
\title{primes}
 \date{}
\maketitle

\section{Problem Statement.}
\paragraph{}
In her quest to rule the world, Meriem now have to face the lord of the dark world.
In order to defeat this monster Meriem must find K prime numbers that sum to the
magic number N. Meriem remembered Goldbach's conjecture and thought that it could help her.
\paragraph{}
your task now is to help her find K primes that sum to N.
\paragraph{}
\textbf{definition of a prime.}
A prime number (or prime) is a natural number greater than 1 that has no positive divisors other than 1 and itself
(like 2,3,5,7,11 ...)
\paragraph{}
\textbf{Goldbach's conjecture.} states that any even number (greater than 2) is a sum of two primes.
$$ \forall\ 2<n\ \ \exists\  p1,p2\ such\ that\  p1+p2=n $$
where n is even and p1 , p2 are two primes. 
this conjecture have been proven to hold for any even number (greater than 2) less than $10^{18}$
\paragraph{}
can you help meriem find K primes such that their sum = N, or say that they dont exist?
\section{Input.}
the only input will be the magic number n and k the number of primes in that order.
$$4\le N \le 2500\ \  and\ \  2 \le K \le 1000.$$

\section{Output.}

print out K prime numbers that sum to N. or if this is not possible print "Impossible" without the quotes.

\section{Examples.}
\subsection{example 1}
Input:
\begin{lstlisting}[language=Python]
2036 4
\end{lstlisting}
Output:
\begin{lstlisting}[language=Python]
509 509 509 509 
\end{lstlisting}
\subsection{example 2}
Input:
\begin{lstlisting}[language=Python]
1473 3
\end{lstlisting}
Output:
\begin{lstlisting}[language=Python]
13 541 919
\end{lstlisting}
\subsection{example 3}
Input:
\begin{lstlisting}[language=Python]
1000 1000
\end{lstlisting}
Output:
\begin{lstlisting}[language=Python]
Impossible
\end{lstlisting}
\subsection{example 4}
Input:
\begin{lstlisting}[language=Python]
15 3
\end{lstlisting}
Output:
\begin{lstlisting}[language=Python]
3 7 5
\end{lstlisting}


\end{document}

