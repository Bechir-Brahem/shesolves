%&latex
\documentclass[10pt]{article}
\makeatletter
\renewcommand{\@seccntformat}[1]{}
\makeatother
\usepackage[utf8]{inputenc}
\usepackage{amssymb}
\usepackage{amsmath}
\usepackage{geometry}
\usepackage{listings}
\usepackage{xcolor}

\definecolor{codegreen}{rgb}{0,0.6,0}
\definecolor{codegray}{rgb}{0.5,0.5,0.5}
\definecolor{codepurple}{rgb}{0.58,0,0.82}
\definecolor{backcolour}{rgb}{0.95,0.95,0.92}

\lstdefinestyle{mystyle}{
	backgroundcolor=\color{backcolour},
	commentstyle=\color{codegreen},
	keywordstyle=\color{magenta},
	numberstyle=\tiny\color{codegray},
	stringstyle=\color{codepurple},
	basicstyle=\ttfamily\footnotesize,
	breakatwhitespace=false,
	breaklines=true,
	captionpos=b,
	keepspaces=true,
	numbers=left,
	numbersep=5pt,
	showspaces=false,
	showstringspaces=false,
	showtabs=false,
	tabsize=2,
    numbers=none
}

\lstset{style=mystyle}

\begin{document}
\title{BAD 7!}
 \date{}
\maketitle
\section{Problem Statement.}
You hate the number 7 right? Me too!\\
Let us count the number of integers without the digit 7 in both \textbf{decimal} (base 10) and \textbf{octal} (base 8).\\
How many such integers are there between 1 and N?
\section{Input}
$$ 1\le N \le 100 $$
\section{Output.}
Print an integer representing the answer.
\section{Examples.}
\subsection{example 1}
Input:
\begin{lstlisting}[language=Python]
20
\end{lstlisting}
Output:
\begin{lstlisting}[language=Python]
17
\end{lstlisting}
Among the integers between 1 and 20, 7 and 17 contain the digit 7 in decimal. Also , 7 and 15 contain the digit 7 in octal.\\
And so , the 17 integers other than 7, 15, and 17 meet the requirements.
\end{document}

