
%&latex
\documentclass[10pt]{article}
\makeatletter
\renewcommand{\@seccntformat}[1]{}
\makeatother
\usepackage[utf8]{inputenc}
\usepackage{amssymb}
\usepackage{amsmath}
\usepackage{geometry}
\usepackage{listings}
\usepackage{xcolor}

\definecolor{codegreen}{rgb}{0,0.6,0}
\definecolor{codegray}{rgb}{0.5,0.5,0.5}
\definecolor{codepurple}{rgb}{0.58,0,0.82}
\definecolor{backcolour}{rgb}{0.95,0.95,0.92}

\lstdefinestyle{mystyle}{
	backgroundcolor=\color{backcolour},
	commentstyle=\color{codegreen},
	keywordstyle=\color{magenta},
	numberstyle=\tiny\color{codegray},
	stringstyle=\color{codepurple},
	basicstyle=\ttfamily\footnotesize,
	breakatwhitespace=false,
	breaklines=true,
	captionpos=b,
	keepspaces=true,
	numbers=left,
	numbersep=5pt,
	showspaces=false,
	showstringspaces=false,
	showtabs=false,
	tabsize=2,
    numbers=none
}

\lstset{style=mystyle}

\begin{document}
\title{That's Magic!}
 \date{}
\maketitle
\section{Problem Statement.}
A magician has the following three cards:
\begin{itemize}
\item A red card with the integer A.
\item A green card with the integer B.
\item A blue card with the integer C.
\end{itemize}
He can do the following operation \textbf{at most K times} (K times or less)
\begin{itemize}
\item Choose one of the three cards and multiply the written integer on it by 2.
\end{itemize}
His magic trick is successful if, after the operations, these conditions are satisfied :
\begin{itemize}
\item The integer on the green card is \textbf{strictly} greater than the integer on the red card.
\item The integer on the blue card is \textbf{strictly} greater than the integer on the green card.
\end{itemize}
Can the magician perform his trick successfully?\\
If the magic can be successful, print “You tricked us!”; otherwise, print ”Oh no!”

\section{Input}
$$ 1\le A \le 10 $$
$$ 1\le B \le 10 $$
$$ 1\le C \le 10 $$
$$ 1\le K \le 20 $$
\section{Output.}
If the magic can be successful, print “You tricked us!”; otherwise, print ”Oh no!”
\section{Examples.}
\subsection{example 1}
Input:
\begin{lstlisting}[language=Python]
7 2 5 3 
\end{lstlisting}
Output:
\begin{lstlisting}[language=Python]
You tricked us!
\end{lstlisting}
The magic will be successful if, for example, he does the following operations:
\begin{itemize}
\item First, choose the blue card. The integers on the red, green, and blue cards are now 7, 2, and 10, respectively.
\item Second, choose the green card. The integers on the red, green, and blue cards are now 7, 4, and 10, respectively.
\item Third, choose the green card. The integers on the red, green, and blue cards are now 7, 8, and 10, respectively.
\end{itemize}


\end{document}

