
%&latex
\documentclass[10pt]{article}
\makeatletter
\renewcommand{\@seccntformat}[1]{}
\makeatother

\usepackage[utf8]{inputenc}
\usepackage{amssymb}
\usepackage{amsmath}
\usepackage{geometry}
\usepackage{listings}
\usepackage{xcolor}

\definecolor{codegreen}{rgb}{0,0.6,0}
\definecolor{codegray}{rgb}{0.5,0.5,0.5}
\definecolor{codepurple}{rgb}{0.58,0,0.82}
\definecolor{backcolour}{rgb}{0.95,0.95,0.92}

\lstdefinestyle{mystyle}{
	backgroundcolor=\color{backcolour},
	commentstyle=\color{codegreen},
	keywordstyle=\color{magenta},
	numberstyle=\tiny\color{codegray},
	stringstyle=\color{codepurple},
	basicstyle=\ttfamily\footnotesize,
	breakatwhitespace=false,
	breaklines=true,
	captionpos=b,
	keepspaces=true,
	numbers=left,
	numbersep=5pt,
	showspaces=false,
	showstringspaces=false,
	showtabs=false,
	tabsize=2,
    numbers=none
}

\lstset{style=mystyle}

\begin{document}
\begin{center}
    \Huge { \textbf{Ons and factorials}}\\
    \normalsize  { input:  STDIN}\\
    \normalsize{    output: STDOUT}
\end{center}
\section{Problem Statement.}
Sandra likes to count the number of consecutive zeros at the end of factorials\\
Given an integer N , return the number of trailing zeroes in N! 
$$ n!= n*(n-1)*(n-2)...1 $$
e.g: $ 5!=5*4*3*2*1=120 $
\paragraph{}
\section{Input}
$$ 1\le N \le 100000 $$
\section{Output.}
the number of trailing zeroes in N!.
\section{Examples.}
\subsection{example 1}
Input:
\begin{lstlisting}[language=Python]
3
\end{lstlisting}
Output:
\begin{lstlisting}[language=Python]
0
\end{lstlisting}

\subsection{example 2}
Input:
\begin{lstlisting}[language=Python]
5
\end{lstlisting}
Output:
\begin{lstlisting}[language=Python]
1
\end{lstlisting}
\textbf{Explanation 1:} 3! = 6, no trailing zero.\\
\textbf{Explanation 2:} 5! = 120, one trailing zero.
\end{document}

