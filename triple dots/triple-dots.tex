
%&latex
\documentclass[10pt]{article}
\makeatletter
\renewcommand{\@seccntformat}[1]{}
\makeatother

\usepackage[utf8]{inputenc}
\usepackage{amssymb}
\usepackage{amsmath}
\usepackage{geometry}
\usepackage{listings}
\usepackage{xcolor}

\definecolor{codegreen}{rgb}{0,0.6,0}
\definecolor{codegray}{rgb}{0.5,0.5,0.5}
\definecolor{codepurple}{rgb}{0.58,0,0.82}
\definecolor{backcolour}{rgb}{0.95,0.95,0.92}

\lstdefinestyle{mystyle}{
	backgroundcolor=\color{backcolour},
	commentstyle=\color{codegreen},
	keywordstyle=\color{magenta},
	numberstyle=\tiny\color{codegray},
	stringstyle=\color{codepurple},
	basicstyle=\ttfamily\footnotesize,
	breakatwhitespace=false,
	breaklines=true,
	captionpos=b,
	keepspaces=true,
	numbers=left,
	numbersep=5pt,
	showspaces=false,
	showstringspaces=false,
	showtabs=false,
	tabsize=2,
    numbers=none
}

\lstset{style=mystyle}

\begin{document}
\title{Triple Dots}
 \date{}
\maketitle
\section{Problem Statement.}
\paragraph{}
Sondes is in charge of the recruitement process in her company. She starts by registering the names of the new members.
However, because of the large number of these members, she decided to cut some corners.
\paragraph{}
When a new member with the name S decides to join, if the length of his name is less than or equal to K, she will write out his/her full name, but if the name is longer than K, she will write the first K characters of the name and append three dots (...).\\
For example, when K = 3 and she receives the name “Ahmed” she will register him as “Ahm...”\\
And when K = 5 and she receives the name “Aycha” she will register her as “Aycha”\\
\paragraph{}
Given K and the name S, find out how X will register the name S.
\paragraph{}
\section{Constraints.}
$$ 1\le K \le 100 $$
$$ 1\le length(S) \le 100 $$
\section{Output.}
Print the string as stated in the problem statement.
\section{Examples.}
\subsection{example 1}
Input:
\begin{lstlisting}[language=Python]
3 Ahmed
\end{lstlisting}
Output:
\begin{lstlisting}[language=Python]
Ahm...
\end{lstlisting}
\end{document}
